\section{{\tool} Algorithms}
\label{algosection}

In this section, we present the key algorithms in {\tool}.
%%ASE
%%to solve the problem formulated in Section~\ref{model}.
The main algorithm (\code{System\-Match}) is for matching the program
entities between two versions of a system, using Definition~1 and
Definition 3 for the matching criteria. This algorithm utilizes a
greedy algorithm (\code{Simple\-Match}) to match the entities in two
given sets. The last algorithm, \code{FindTestCase}, uses the list of
matched entities to find existing test cases that need~fixing.

%We first start with the algorithm to calculate
%the common subset of two sets of entities as in Definition 5, which
%can be used to detect the matching between the entities in any two
%sets. Then, we present our algorithm to match the entities between two
%versions of the system for feature tracking.

% which selects the
% candidates and calculates the common subset via the first algorithm.

%The pseudo-code for algorithms is in Figure~\ref{matching}.

\subsection{Key Design Ideas}

Initially, one might think that matching the entities could be done
using a classical bipartite matching algorithm following
Definition~1. However, such a (brute-force) algorithm will not work
because of the following issues:

1. {\em Efficiency}. Classical bipartite matching has the
   computational complexity of $O(n^3)$, with $n$ being the number of
   entities. With this high level of computational complexity, the
   algorithm does not scale well for large systems with several
   thousands of entities (classes and methods).

2. {\em Determinism}. The criteria for matched entities is recursive
   by nature. Assume that $u$ and $v$ are two related entities. To
   match $u$ to $u'$ (Definition~1), we might need to compute the
   similarity of two sets including $v$ and some entity $v'$ related
   to $u'$ (Definitions~3 and 4). This computation requires the
   matching information of $v$ to $v'$ (Definition~5). If $v$ and $v'$
   are not yet matched, this matching computation in turn requires the
   matching of $u$ and $u'$, which is currently under calculation.


%uses the knowledge from already-matched code entities to incrementally
%update the similarities of not-yet-matched entities for further
%matching.

To overcome these two issues, we have designed a novel algorithm for
this matching problem. The algorithm works in an {\bf iterative}
fashion, using \emph{already-matched} entities to find new candidates
and compute their similarity for further matching. That is, instead of
considering all pairs of entities as possible matched pairs, {\tool}
maintains and processes a list of pairs as candidates. To compute the
similarity of two entities in a candidate pair, {\tool} uses only the
already-matched entities (\eg in finding the common subset in
Definition~5), disregarding the recursive nature of the similarity
calculation. If they are matched, their related entities
(\eg methods, callers, callees, or sub-classes) are potentially
matched entities, thus, are considered to be added to the list of
candidates (via re-calculating their similarity). All the remaining
entities are compared pairwise at the final step, which is much less
expensive than the pairwise comparison at the beginning.

At first, {\tool} uses name-based and implementation-based criteria to
find the initial candidates. That is, it initializes its candidate
list with the entities having the same simple names (class or
method names without the signatures), and then the entities with
similar bodies. To find the entities having similar bodies, {\tool}
utilizes locality-sensitive hashing (LSH)~\cite{lsh} on the
characteristic vectors extracted from their implementation code. For
example, the body vector of a class is a binary vector of the
occurrences of classes used in its fields and
methods. Locality-sensitive hashing is a hashing scheme that provides
the same hash code for similar vectors. Thus, to find entities with
similar implementation code, {\tool} hashes their vectors and only
compares the entities having the same hash code.

Next, we will discuss the details of three algorithms to find the
matched subset of two entity sets (Definition 5), to match the
entities between two versions (Definition 1), and to detect the test
cases that need adaptation.

\subsection{Simple Matching}
\label{simplematch}

\begin{figure}
\begin{lstlisting}[stepnumber=1,numbers=left,numbersep=-5pt]
		function match($U$, $U'$, $N$)
		//find matched set $U \otimes U'$, output as a map $N$
			$N$.clear()
			repeat
				$L$ = new descendingly sorted list associating with $sim$
				for all $(u,u') \in U \times U'$
					if($sim(u,u') \geq \sigma$)
						$L$.add$(u,u')$ //collect all pairs of candidates
				if $L = \emptyset$ break //no pair collected => done
		
				while($L \neq \emptyset$)
					$(u,u') = L$.top() //match top ones
					$N$.add$(u,u')$
					remove all $(u,v') \in L$ and $(v,u') \in L$ //remove their pairs
			until false
\end{lstlisting}
\caption{Simple Matching Algorithm}
\label{cmatching}
\end{figure}

Let us first describe the algorithm to match the entities between two
sets of entities $U$ and $U'$ in two different versions
(Definition~5). It is called \code{simple match}, because it
is a separate computation step for
%simplifies the computation of
the similarity of two entities without recursion. That is, for each $u
\in U$ and $u' \in U'$, if $u$ and $u'$ are already matched, then
$sim(u,u') = 1$. Otherwise, it initializes $sim(u,u')=nsim(u,u')$ with
$nsim$ being their name similarity defined in Definition 2 and updates
this value if more information is available. Thus, $sim$ is always
defined for any pair of entities $u$ and $u'$. In addition, it uses a
greedy approach that iteratively selects the matching pairs
descendingly by their similarity degrees.

%, thus, reducing the complexity to $O(n^2)$.

Fig.~\ref{cmatching} illustrates the algorithm. First, it
collects all pairs of entities $u \in U$ and $u' \in U'$ satisfying
the necessary condition for matching, \ie$sim(u,u') \ge
\sigma$. Those pairs are maintained in a heap, \ie a descending
sorted list $L$ (lines 5-8), sorted by the similarity
$sim(u,u')$. Then, {\tool} repeatedly detects the pair $(u,u')$ with
the highest $isim$ in that sorted list as a matched pair (lines
12-13), and removes all pairs containing either one of its two
entities $u$ or $u'$ (line 14). This step ensures that each entity $u$
is matched at most one other entity $u'$ in which the interaction
similarity $isim$ is maximal. The newly matched pairs now are used
to recalculate $sim$ for the non-matched entities in $U$ and $U'$
when the process repeats (lines 4-15) until no other candidates
can be collected (line 9).

%This greedy approach is reasonable in this situation because we want to map each entity in one version with at most one entity in the other version of the system and there are no two entities having the same signature in one version of a system.

\subsection{System Matching}


Fig.~\ref{matching} illustrates the main algorithm to match entities
from a version $P$ to a new version $P'$. It outputs a map $C$
containing pairs of matched classes $(c,c')$ and a map $M$ containing
pairs of matched methods $(m,m')$. This algorithm has two main steps:
matching the classes and then the methods. For those steps, the same
procedure is applied: first, it processes for the
candidates with the same simple name, then for the candidates having
similar characteristic vectors extracted from body code, and finally,
all remaining candidates are compared pair-wise. The candidates with the
same names or similar vectors are identified via LSH.

\begin{figure}
\begin{lstlisting}[stepnumber=1,numbers=left,numbersep=-5pt]
		function track($P, P'$)
		//match entities in two versions $P$ and $P'$ to two maps $C$ and $M$
			$C$ = matchClasses()
			$M$ = matchMethods()
		return $C, M$
		
		function matchClasses()
			$X$.addAll(classes of $P$) and $X'$.addAll(classes of $P'$)
			$N$ = HashbyName($X$) and $N'$ = HashbyName($X'$)
			for each $n$ in $N$.keys $\cap$ $N'$.keys: process($N(n), N'(n), C$)
			$H$ = LSHbyBodyVector($X$) and $H'$ = LSHbyBodyVector($X'$)
			for each $k$ in $H$.keys $\cap$ $H'$.keys: process($H(k), H'(k), C$)
			process($X, X', C$)
		return $C$
		
		function matchMethods()
			$X$.addAll(methods of $P$) and $X'$.addAll(methods of $P'$)
			for each matched classes pair $(c,c') \in C$:
				process($c$.methods, $c'$.methods, $M$)
			$N$ = HashbyName($X$) and $N'$ = HashbyName($X'$)
			for each $n$ in $N$.keys $\cap$ $N'$.keys: process($N(n), N'(n), M$)
			$H$ = LSHbyBodyVector($S$) and $H'$ = LSHbyBodyVector($S'$)
			for each $k$ in $H$.keys $\cap$ $H'$.keys: process($H(k), H'(k), M$)
			process($X, X', M$)
		return $M$
		
		function process($U, U', O$)
		//match entities in $U$ and $U'$, add them to the map of matched entities $O$, and remove them from the remaining sets.
			match($U$, $U'$, $N$)
			$O$.addAll($N$)
			$X$.removeAll($N$.keys)
			$X'$.removeAll($N$.values)
\end{lstlisting}
\caption{System Matching Algorithm}
\label{matching}
\end{figure}

\subsubsection{Matching Classes}

{\tool} matches the classes (lines 7-14) before matching the methods
(lines 16-25). To match classes, it first processes the candidates
having the same simple names (lines 9-10). It uses string-based
hashing scheme to hash the classes into buckets, each bucket is a
set of entities having the same name. Then, for each pair of buckets
for the same name, it uses \code{simple match}
(Section~\ref{simplematch}) to match their entities. The matched
entities are added to the global map $C$ and removed from the sets of
remaining entities $X$ and $X'$.

Then, {\tool} processes the candidates having sufficiently similar
vectors extracted from their bodies (lines 11-12). The process is
similar to the previous step. The vector extracted from a class $c$
describes the classes used in $c$. Instead of string-based hashing,
{\tool} uses locality-sensitive hashing to hash and find similar
vectors. Finally, {\tool} matches the remaining classes pairwise (line
13).

Note that, the similarity of names or body code is used only to
suggest the candidates for matching. Those candidates are actually
matched using the interaction similarity (defined in Definition~3) as
in the algorithm in Section~\ref{simplematch}.

\subsubsection{Matching Methods}

To match the methods, the algorithm first finds the candidate methods
in the classes that are matched in the previous step (lines 18-19)
because if the classes are matched, their methods should be the
candidates for matching. However, there are many cases in which the
methods might be extracted/split from the old class. Thus, the next
step of the algorithm is to find the candidate methods {\em across the
classes}. Once again, {\tool} uses name and body to suggest the
candidates. It starts processing the candidate methods in any classes
having the same names (lines 20-21). Then, it considers the methods
having similar bodies (lines 22-23). The body vector of a method is a
binary vector of the occurrences of methods that are called in its
body, \ie the vector representing the set \code{Callee}. Finally, the
remaining methods are pairwise compared (line 24). Similar to the
classes, candidate methods are matched using the algorithm in
Section~\ref{simplematch} with the interaction similarity criteria in
Definition~1.

%\begin{figure}
%\begin{lstlisting}[basicstyle=\small\sffamily, stepnumber=1,numbers=left, language = Delphi]
%function FeatureTrack($P, P'$)
%// Match entities in two versions $P$ and $P'$ to a map $M$
%//  Step 1.1. Match classes using name for candidates
% $C$.addAll(classes of $P$) and $C'$.addAll(classes of $P'$)
% $H$ = HashbyName($C$) and $H'$ = HashbyName($C'$)
% for each name $n$ in $H$.keys $\cap$ $H'$.keys //candidates by name
%  match($H(n), H'(n), N$) and update($N, M_C, C, C'$)
%
%//  Step 1.2. Match classes using body for candidates
%//  use locality-sensitive hashing: similar body => same hash code
% $H$ = LSHbyBodyVector($C$) and $H'$ = LSHbyBodyVector($C'$)
% for each hash code $k$ in $H$.keys $\cap$ $H'$.keys
%  match($H(k), H'(k), N$) and update($N, M_C, C, C'$)
%
%//   Step 1.3. Match all remaining classess
% match($C, C', N$) and update($N, M_C, C, C'$)
%
%//  Step 2.1. Match methods using name for candidates
% $M$.addAll(methods of $P$) and $M'$.addAll(methods of $P'$)
% for each matched classes pair $(A,A') \in M_C$
%  match($A$.methods, $A'$.methods, $N$) and update($N, M_M, M, M'$)
% $H$ = HashbyName($M$) and $H'$ = HashbyName($M'$)
% for each name $n$ in $H$.keys $\cap$ $H'$.keys //candidates by name
%  match($H(n), H'(n), N$) and update($N, M_M, M, M'$)
%
%//  Step 2.2. Match methods using body for candidates
% $H$ = LSHbyBodyVector($M$) and $H'$ = LSHbyBodyVector($M'$)
% for each hash code $k$ in $H$.keys $\cap$ $H'$.keys
%  match($H(k), H'(k), N$) and update($N, M_M, M, M'$)
%
%//  Step 2.3. Match all remaining methods
% match($M, M', N$) and update($N, M_M, M, M'$)
%
%function update($N, O, U, U'$)
%//remove newly-found matched entities $N$ from unmatched entities $U$ and $U'$ and add them to global matched entities $O$
% $O$.addAll($N$)
% $U$.removeAll($N$.keys)
% $U'$.removeAll($N$.values)
%
%\end{lstlisting}
%\caption{Entities Matching Algorithm}
%\label{matching}
%\end{figure}

\subsection{Recommending Repairing Broken Method Calls}
{\tool} uses a simple criterion to find the previously run test cases
that require an repair. That is, a test case $t$ is recommended for repair if it uses an entity $u$ (\eg declares an object of class $u$ or calls method $u$) that needs to be replaced. {\tool} considers $u$
to be replaced if:

1. It is not matched to any entity in the new version (\ie is deleted), or

2. Its matched entity $u'$ has a different signature.

Using this criteria, after matching all entities for the system,
{\tool} analyzes all the existing test cases. If a test case
$t$ satisfies the aforementioned criterion with some entities $u$, {\tool} will output $t$, $u$, and its matched entity $u'$ (if any) as the recommendation, \ie $t$ is recommended for repair in which $u$ is replaced by $u'$.

%(In the future works, we plan to use more sophisticated change
%analysis techniques to find the test cases that is affected by the
%change of the system). 
