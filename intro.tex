\section{Introduction}


During development, a software system constantly evolves. After changes
are made to the system, regression testing is often performed to help
detect the regression faults in the previously existing
functionality. Developers often run the new version of the system
against its existing test cases to make sure that those changes did
not introduce unintended defects. However, as the system evolves, the
program entities such as classes and methods that realize its
functionality are also modified or replaced by other entities. Therefore,
in regression testing, some previously run test cases need to be
adapted to use the replacing classes and methods.

%in accordance with such changes in program entities.

For example, at version v0.9.20 of jFreeChart, an open-source chart
drawing tool in Java,
\code{DatasetUtilities.getDomain\-Extent(Dataset)} ($m$) is
responsible for the calculation of the domain of input data
points. Several changes were made to that code to create v0.9.21. A
new method \code{DatasetUtilities.findDomainExtent(Dataset)} ($m'$)
with a different implementation was created to replace $m$, which was
deprecated according to jFreeChart's documentation. All existing test
cases at version v0.9.20 that contain method calls to $m$ had to be
changed with respect to this replacement of $m$. That is, all method
calls to $m$ had to be redirected to $m'$ to ensure that those test
cases would be able to test the same aspects as before in the system.

During software development with several distributed team members, the
documentation for such replacements of classes/methods is not always
available in time. It is helpful for the testers, who might not be the
authors of those entities, to have an automated tool that provides
such matching of the entities between two versions. From the knowledge
about the matching, developers can fix the broken method calls in
existing test cases.
%%proceed faster the adaptation of existing test cases with the use of
%%replacing entities.

%The testers usually have to manually figure out by themselves such
%replacements by tracking those classes and methods across
%versions. That task is very tedious and time-consuming.  Therefore, it
%is desirable to have an automatic tracking tool that supports the
%matching of the entities between two versions, i.e. recovers the pairs
%of entities with the same or slightly changed role/functionality.

%Unfortunately, during the development process, the documentation for
%code changes and specifically for such classes/methods' replacements
%is not always complete. The testers usually have to manually figure
%out by themselves such replacements by tracking those classes and
%methods across versions. That task is very tedious and time-consuming.
%Therefore, it is desirable to have an automatic tracking tool that
%supports the matching of the entities between two versions,
%i.e. recovers the pairs of entities with the same or slightly changed
%role/functionality.

%The previously run test case for the data domain calculation at
%version v0.9.20 had to be adapted accordingly with the replacement of
%$m$ because the body of that test case contains the method invocations
%to $m$. That call ought to be redirected to $m'$ to ensure that those
%test cases would be able to test the same aspects as before in the
%system.

%Software systems in reality continuously evolve. Regression testing is
%used to check whether the change of a system does not affect some of
%its functionality. Since when a software system changes, its program
%entities, such as classes/methods, also change. Thus, to perform
%regression testing, one needs to match entities in the old version to
%ones in the new version that have the same role or provide the same
%functionality. For example, if a method is renamed from $m$ to $m'$,
%developers should replace any calls to $m$ by $m'$ in the test cases
%to ensure those test cases will be able to test the same aspects of
%the system.

%A program always consists of logical entities such as classes,
%methods, variables, and so on. One or multiple program entities are
%designed to provide a functionality in the system. When the program
%evolves for bug fixing, code restructuring, or feature enhancing, its
%entities are also changed (being renamed, moved, or modified). Those
%changes are of interest for system comprehension, maintenance, and
%other tasks.

In general, such a tool must match an entity in an old version to the
corresponding entity with the same or slightly changed functional role
in the new version. Intuitively, program entities across two versions
could be matched based on their names as the names could reveal the
roles or functionality. One could also match via their implementations
as the implementation describes functionality. However, in reality,
there are cases in which the classes/methods are changed in both
names and internal implementations, making such matching difficult.

%The lack of documentation and the large number of program entities in
%large-scale systems makes manual matching/tracking of those entities
%tedious and time-consuming. Therefore, it is desirable to have an
%automatic tracking tool supporting the matching the entities between
%two versions, i.e. recover pairs of entities that support the same or
%slightly changed functionality. Intuitively, we could match entities
%based on their names (as naming could reveal the role or
%functionality). We could also match based on implementation (as
%implementation describes functionality). However, in reality, there
%are some cases, the entities might change both in name and
%implementation, making the such matching for them difficult.
%as shown in the next example, 

In this paper, we propose {\tool}, an {\em interaction-based} tool to
match the program entities mainly based on the interaction
%and dependencies 
among them. Our idea is that, an entity is generally associated
with some role/functionality in the system, and this role is rather
stable as the system evolves. The role of an entity correlates with
its interaction
%/dependencies 
with other entities (\eg how it uses or is used by others). Thus,
{\tool} matches the entities between versions based on the similarity
of their interaction.
%/dependencies. 
Based on the matched entities, {\tool} then identifies the
existing test cases that need to be changed to fix broken method calls
by using the replacing entities.
%use the replacements of old entities.

%This is very useful because without it, a tester must go through all
%previous test cases and determine if they requires an adaptation due
%to the entities' replacements.

% Representation of system

{\tool} represents a system at any version as {\em an attributed,
  directed multigraph}, in which the {\em nodes represent classes and
  methods}, the {\em edges represent different kinds of their
  interaction}, and the {\em attributes represent the associated
  information} of the entities, \eg their names, signatures,
etc. {\tool} supports tracking two types of program entities: classes
and methods. The attributes and relations of classes/methods represent
three main kinds of their interaction in a program: {\em method
  invocation}, {\em data access}, and {\em inheritance}.

For each entity, {\tool} maintains three kinds of attributes and
relations/interaction: {\em signature}, {\em inheritance}, and {\em
  collaboration}. The signature of an entity contains its name and
interface information. The attributes of inheritance relations of an
entity refer to the other classes that inherit or are inherited from
that class. The attributes of collaboration relations of an entity
refer to the entities that are used in its body or use that entity in
their bodies.  Thus, for each entity, {\tool} associates it with {\em
  an interaction set} corresponding to each kind of
interaction. Matching of program entities is then formulated as the
matching between the nodes of two graphs at two versions, which takes
into account the similarity of interaction of the entities as one of
the matching criteria.

The key challenge of computing interaction similarity between two
entities is that it has a recursive nature. For~example, when
matching two entities $A$ and $A'$, {\tool} must~compare two other
entities $B$ and $B'$ that are used/called respectively within $A$
and $A'$. Such comparison of $B$ to $B'$ in~turn~requires the matching
of $A$ and $A'$ because $A$ and $A'$ use/call $B$~and $B'$
respectively. Another case of this nature is when $m$~calls~$n$ and
both $m$ and $n$ are renamed.
%----------------------------
%{\tool} calculates the interaction similarity between two entities as
%a weighted sum of the similarity of all of their interaction sets. The
%similarity for two interaction sets is defined as the ratio between
%the size of their common subset and the average of their sizes where
%the common subset is {\em the maximum set of the pairs of
%already-matched entities} between two sets.
%------------------------------
To address that, {\tool} differs from the existing approaches in its
efficient measurement of the interaction similarity and matching of
the entities in which two key advances are combined: 1) {\em
  interaction-based} and 2) {\em iterative matching}. First, the
candidates are discovered based on interaction, \ie when two entities
are matched, the entities related to them, such as the containing
classes, superclasses, callers, and callees, will be the candidates
for further matching. Second, the entities are matched iteratively,
\ie the already-matched entities are used to calculate and update the
interaction similarities of the remaining candidates. {\tool} also
uses other similarity measures in traditional origin analysis (\eg
name, code structure, and calling relation similarity) to reduce the
number of comparisons.

Our empirical evaluation shows that {\tool} can achieve 84--99\%
accuracy in identifying the previous test cases that need to be
repaired with regard to the replacements of entities and provide the
matching of entities to fix broken method calls in those test
cases. The key contributions of this paper include

1. A novel formulation and algorithm for matching the program entities
having the same or slightly changed roles/functionality across two
versions of an evolving system;

%2. A technique that uses the matching algorithm to determine the
%previously run test cases that need to be adapted accordingly to the
%classes/methods' replacements. The resulting matches from the matching
%algorithm will facilitate the test case adaptation in regression
%testing;

2. A method that uses the resulting matches 
%from the matching algorithm 
to determine the existing test cases that need to be repaired and to
recommend the fixes to their broken method calls in the test cases;

% in the existing test cases,

%adapted to use replacing entities and facilitate the test case
%adaptation in regression testing;

3. {\tool}, an interaction-based entity matching tool; and

4. An empirical evaluation on real-world systems, showing the
   correctness and scalability of our tool.

Sections~2 presents motivating examples of entity matching for
regression testing. Section~3 and Section~4 present our formulation
and algorithms. Sections~5 describes the evaluation. Section~6 is for
related work. Conclusions appear last.
