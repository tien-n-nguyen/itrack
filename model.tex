\section{Formulation}
\label{model}

\subsection{Key Concepts}

This section presents our model to formulate the tracking problem of
program entities.
%%ASE
%% for adapting regression test cases. 
In our model, a \emph{system} is represented as a set of
{\bf program entities} (\eg classes, methods) with the
{\bf attributes} (\eg name/interface information), and the
%%ASE
%%\emph{relations} and 
{\bf interaction} among those entities (\eg inheritance, invocation,
and collaboration). \emph{Entity tracking} is applied by matching each
entity in the previous version to at most one entity in the next
version. The matching is based on the \emph{similarity} between their
entities. Such similarity is calculated via the weighted sum of the
similarity values of the {\em attributes} and the \emph{interaction}
%%and relations
associated with the entities. 
%
Next, we will describe in detail the key concepts in {\tool}, which
was adapted and extended from the program differencing tool,
iDiff~\cite{ase11-tool}.

%
%entities, attributes, interaction, and the formulation of the matching
%conditions as well as the similarity measurements.

% between entities.

%\subsection{Entity, Attribute, Relation and Interaction}
\subsection{Entity, Attribute, and Interaction}

%Each entity has a set of attribute for signature, inheritance and collaboration attributes for interaction with other entities.

{\tool} is interested in tracking two types of entities: classes
and methods. The attributes
%%ASE
%%relations, 
and interaction of the entities reflect three main kinds of
interaction among entities in a system: method
invocation, data access, and inheritance. {\bf The attributes
and interaction of a class $C$ include}:

\begin{itemize}

\item {\bf Its signature}, containing both the package's and the
  class names. The term \emph{simple name} is used to refer to the
  class name. For example,
  \code{jfreechart.chart.labels.XY\-ItemLabelGenerator} is the
  signature of a class. The corresponding simple name is
  \code{XYItemLabelGenerator};

\item {\bf Inheritance}: $C$ has the attribute
  \code{SuperClass}($C$) to represent a set of classes from which it
  inherits (super-classes), and the attribute \code{SubClass}($C$)
  for the classes/interfaces that inherit/implement $C$ (sub-classes);

\item {\bf Collaboration}: $C$ has the attribute
  \code{Container}($C$) that is the set of classes which declare $C$ as
  a type of their field(s) or have $C$ in their methods'
  signatures. The attribute, \code{Containee}($C$), is the set of
  classes having $C$ as one of their containers, \ie the classes that
  are declared as the types of fields of $C$ or are in one of the
  methods' signatures of $C$. The attribute \code{User}($C$) contains
  the methods that use $C$ in their bodies.

\end{itemize}

\noindent For Java, we consider an interface as an abstract class. {\bf The
attributes and relations/interaction of a method $m$ include}:

\begin{itemize}

\item {\bf Its signature}, containing the package, class, and method's
names, the list of parameter types and the return type. The term {\em
simple name} is used to refer to the combination of class and
method's names in the form of \code{Class.Method}. For example,
\code{DragNDropTool.setDragOn} is the simple name of the method in
Example 2;

%For example here is an signature: \code{long org.jfree.data.xy.XYDatasetTableModel.datasetChanged(long ?org.jfree.data.general.DatasetChangeEvent)} [Tung: please provide a better example]

\item {\bf Inheritance}: $m$ has two attributes of this type.
\code{Overrider}($m$) is the set of methods that override $m$, and
\code{Overridee}($m$) includes the ones that $m$ overrides;

\item {\bf Collaboration}: $m$ has three attributes of
this type. \code{Caller}($m$) contains the methods that call $m$,
while \code{Callee}($m$) is the set of methods that are called within
$m$'s body. \code{Local}($m$) is the set of classes that are used in
the body of $m$ (\ie local types).

\end{itemize}

In {\tool}, the entities, their attributes, relations, and
interaction are represented as an attributed, directed multi-graph,
called {\bf Interaction Graph} (IG) in which

- A node represents a class or a method. The label of each node is the
  corresponding entity's signature;

- Each type of interaction existing between two entities is
  represented as an edge connecting two nodes.

\subsection{Tracking and Similarity}

Each version of the system is represented by an interaction
graph. Entity tracking between two versions aims to match each entity
in one graph of a version to another entity in the other graph of the
later version. 
%Our assumption is that the role of entities and their
%interaction within the system are less likely to change than
%their names or concrete implementations through evolution. 
To track an entity during evolution, we could rely on its role and
interaction with other entities in the system, in addition to names or
implementation. Thus, our matching is mainly based on the similarity
of interaction of the entities, while also takes other measurement
criteria into account to reduce the number of comparisons among
entities.

%enrich the sets of candidates.

\begin{Definition}[{\bf Matched entities}]
An entity $u$ in version 1 is matched to an entity $u'$ in version 2 of the system, denoted by $u \equiv u'$, if their interaction similarity, denoted by $sim(u,u')$, is maximal and is sufficiently large with respect to a pre-defined
threshold, \ie $sim(u,u') \rightarrow max$ and $sim(u,u') \geq
\sigma$.
\end{Definition}

Because an entity from version 1 must be matched to at most one entity
in version 2, {\tool} chooses the matched entities with highest
similarity of interaction. In the case that an entity has the same
interaction similarity with more than one other entities, the
similarity in their names will be used.

\begin{Definition}[{\bf Name similarity}]
The name similarity between two entities, denoted by $nsim(u,u')$, is
measured as the ratio between the length of the longest common
sub-sequence (LCS) and the average length of their simple names.
\end{Definition}

To compute name similarity, {\tool} first breaks two names into
two sequences of tokens based on Hungarian or CamelCase notations,
then applies LCS on two sequences. For example,
$nsim$(\code{setDragSourceActive},\code{setDragOn}) = 2/3.5 = 0.57.

\begin{Definition}[{\bf Interaction similarity}]
The interaction similarity between two entities in two versions is the
weighted sum of the similarity of all of their interaction sets and is
computed~as
$$sim(u,u') = \sum_i{\alpha_i*ssim(I_i(u),I_i(u'))}$$ where
$\alpha_i$s are the non-negative weights such that $\sum_i{\alpha_i} =
1$.
\end{Definition}

In this formula, each set $I_i(u)$ is an {\em interaction set}
of~$u$. It is a set of entities corresponding to one kind of
interaction associated with $u$. Entity $u$ can have multiple
interaction sets. 

If $u$ is a class, its interaction sets include \code{SuperClass}($u$),
\code{SubClass}($u$), \code{Containee}($u$), \code{Container}($u$) and
\code{Users}($u$). If $m$ is a method, its interaction sets include
\code{Caller}($m$), \code{Callee}($m$), \code{Local}($m$),
\code{Overrider}($m$), and \code{Overridee}($m$).  Each of such
interaction for $u$ will be compared with the corresponding
interaction set for $u'$ in the new version. To compare two sets $P$
and $P'$, we define $ssim(P,P')$ as follows:

%is the similarity measurement of two sets $P$ and $P'$, which is
%defined as follows:

\begin{Definition}[{\bf Set similarity}]
The similarity between two sets of entities $P$ and $P'$ is the ratio
between the number of their matched entities and the total number of their
entities:
$$ssim(P,P') = \frac {2*|P \otimes P'|} {|P| + |P'|}$$
\end{Definition}

$P \otimes P'$ denotes the {\em matched subset} of $P$ and $P'$,
covering all matched entities in $P$ and $P'$. Definition~4 means
that $P$ and $P'$ will be more similar if they contain more matched
entities. Formally, the matched subset is defined as

\begin{Definition}[{\bf Matched subset}]
The matched subset of two sets of entities $P$ and $P'$ is the list of pairs of their matched entities, i.e. $P \otimes P' = \{(u \in P, u' \in P') |u \equiv u\}$.
\end{Definition}

%The intuition of this definition is that, the larger the number of
%common entities between $P$ and $P'$ is, the more they are
%similar. Conventional wisdom would choose the set of common entities
%as the intersection set of $P$ and $P'$. However, $P$ and $P'$ are
%sets of entities between two versions, which might have changed
%entities. Therefore, to {\em take the changes of entities into
%consideration}, we define the {\em common entities of $P$ and $P'$} as
%the following

Let us take Fig.~\ref{drag} to show an example of the similarity
between two sets of $I$=\code{Callers}(\code{setDragSourceActive}) and
$I'$=\code{Callers}(\code{setDragOn}). Assume that, we could match
\code{active} to \code{active}, and \code{deactivate} to
\code{deactivate} in two versions of class \code{DragNDropTool}, then
their matched subset contains those two pairs, $ssim(I,I')$=
(2*2)/(2+2) = 1.0.

%Assuming that at the time of computing for these two sets, {\tool}
%already had the mapping between the classes that have same names, then
%$ssim(I_1,I_2)=2/6=0.33$.
%Now given that more matchings are found between the classes whose
%names contain \code{ItemLabel} and \code{ToolTip}, then
%$ssim(I_1,I_2)$ could reach 1.0.

%%%This matching problem is really the problem of finding maximum
%%%cardinality bipartite matching of $P$ and $P'$, using $sim(u,u') \geq
%%%\sigma$ as a condition to make the edges.

%To avoid recursive computation, we develop an approximate, iterative algorithm, which will be explained next.

%- A system is represented as an attributed, interaction/dependency graph:
%
%+ Each node represents an entity in the system: package, class/interface, method.
%
%+ A node has a set of attributes: interface (pathname/signature), implementation (body usage vector).
%
%+ Edges represent different relations between nodes: interaction/dependency caller-callee (client-server) - employer-employee/co-operate/collaborator, colleague
%
%X and Y are employer-employee related: Y is used in X, used = inherited, called, declared as field/variable/parameter.
%
%X and Y are colleague related: are both used inside Z' implementation.
%
%Philosophy/Assumption: evolved entities might have changes in interface or implementation but little in interaction. Especially interaction is the most stable (changes at design level).
%
%Thus, similarity on interface/implementation/interaction of entities could be used to tracking them.
%
%Define: sim(x,y) = total similarity
%
%        or matched(x,y) = some conditions...
%
%Problem: graph matching, maximize number of matches.
%
%\subsection{Similarity measurement}
%
%This section discusses the similarity measurements designed for different types of software entities (package, class, method).
%
%1. Method similarity
%
%- signature: return type + name + list of parameter types
%
%- internal interactions:  usages of other methods within its implementation
%
%- external interations: its usage within other methods' implementations
%
%- local types: usages of classes within its implementation
%
%2. Class similarity
%
%- signature: name
%
%- inheritance/interface relation
%
%- declared classes
%
%- body: synthesized from all normalized contained methods' features. All methods' vectors are normalized to have the length of 1 and summed up.
%
%3. Package similarity
%
%- signature: name and containg package's name
%
%- body: all contained packages and classes
%
%4. Name similarity measurement: Names are tokenized based on naming convention and then compared using long common sub-sequence
%
%5. Structure vector similarity: ratio between shared structural features (including the frequency) and total structural features.
%
%6. Interaction set/vector similarity: set similarity: $$sim(X,Y) = \frac {|X \cap Y|} {|X \cup Y|} $$, or distance (normalized, or Spearman correlation??? )???
