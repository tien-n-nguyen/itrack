\section{Related Work}

\vspace{0.05in}
\noindent {\bf Origin Analysis.} Related research to {\tool} is {\em
origin analysis}~\cite{godfrey02,sungkim-wcre05,godfrey05}, which aims
to determine the original version of a program entity as software
changes. The origin analysis tool from Kim, Pan, and Whitehead
(KPW)~\cite{sungkim-wcre05} uses various similarity factors for
matching two versions of the methods. The factors include the methods'
names and signatures, caller/callee sets, textual contents, and
complexity metrics. Comparing to KPW tool, {\tool} has several crucial
advances. First, while comparing caller/callee sets of two methods in
two versions, {\em KPW tool relies on name similarity of methods} in
those sets. Thus, when both caller and callee are renamed, its
matching would fail. {\tool} still works because it is based on the
interaction similarity and on previously matched entities via
iterative matching. Second, in addition to interaction-based matching,
{\tool} is more advanced than KPW in {\em iterative matching}: {\tool}
uses the knowledge from already-matched code entities to incrementally
update the similarities of not-yet-matched entities for further
matching. Therefore, {\tool} can rely on simple matching cases to
resolve more complex matching such as the cases of renamed
callers/callees. Third, {\tool} considers also inheritance and data
relations, thus, can recover class matching, while KPW does not.
Finally, to compare the methods' bodies, because KPW tool uses
token-based clone detection in CCFinder~\cite{ccfinder}, it would
often match a method $m$ to its deprecated one, rather than its
replacement. {\tool} uses interaction similarity to match to the
replacement, thus, provides better matching.

%in adapting regression test cases.

%{\tool} uses internal calling structure that capture better usage
%similarity between methods.

Another related work is BEAGLE~\cite{godfrey02}. To map two methods
across two versions, it first computes internal similarity based on
the metrics like Cyclomatic, S-, and D-Complexity. Then, it considers
the ``calls'' and ``is-called-by'' relationships among two methods and
then computes the similarity of calling structures between methods.
%in which {\em it considers the methods with the same names in both
%  versions as the matched methods}. 
During computing the calling structure similarity, it first finds
the methods with the name matching in two versions and then 
%From each {\em name-matched} method, it 
examines the callee methods and uses other name-matched methods to
detect other matching methods. {\tool} directly computes the
interaction similarity for matching two versions of a method.
%
%rather than name similarity. 
Moreover, {\tool} also analyzes different kinds of interaction
including usage sets of the methods, helping it to deal with the cases
of renaming or refactored. In a later work, Godfrey and
Zou~\cite{godfrey05} use origin analysis in BEAGLE~\cite{godfrey02}
and consider refactoring patterns using calling relations. They also
detect merged and split entities. We could improve {\tool} with this feature.

%This is the current limitation of {\tool}.

%Because BEAGLE is not quite well-suited for feature tracking due to
%its name-based mapping approach, therefore, the result of their
%technique for detecting merged and split entities could be affected.



%{\tool} further analyzes the usage sets of those methods, despite
%their name matching. Thus, {\tool} still works when methods are
%renamed/refactored.

%Another related work is BEAGLE~\cite{godfrey02}. To map two
%methods across two versions, it first computes internal similarity
%based on the metrics like Cyclomatic, S-, and D-Complexity. Then, it
%computes the similarity of calling structures of two methods in which
%{\em it considers the methods with the same names in both versions as
%the matched methods}. From each {\em name-matched} method, it examines
%callee methods and uses other name-matched methods to detect other
%matching methods. {\tool} fundamentally differs from BEAGLE in that it
%uses {\em the interaction similarity for matching two versions
%of a method}, rather than name similarity. {\tool} further
%analyzes the usage sets of those methods, despite their name
%matching. Thus, {\tool} still works when methods are
%renamed/refactored.

Kim and Notkin~\cite{Kim07}'s LSdiff uses a rule inference algorithm
to capture code changes at method-headers. Their later
work~\cite{Kim09} infers complex rules to describe changes to
classes/fields/methods, their containment relationships, and
structural dependencies. Before the rule inferencing step, the first
step (entity matching) in their tool is similar to KPW tool. Their
resulting rules could be improved if LSdiff runs on the mapping
results from our tool. Vdiff~\cite{kim10} outputs syntactic changes in
a Verilog program and provides position-independent differencing.
However, it cannot handle the cases that methods are both relocated
and changed.

% in term of internal implementations.

%structure diff, calling structure diff

\vspace{0.05in}
\noindent {\bf API and Framework Adaptation.} {\tool} is related to
API and framework adaptation approaches~\cite{semdiff08,catchup05}
that recommend the adaptation changes to client code of some APIs when
those APIs evolve. SemDiff~\cite{semdiff08} detects replaced API
methods in a framework. First, the methods in two versions of the
framework's code are mapped based on code structure similarity. Via
those methods' bodies, SemDiff determines their callees. Then, it
mines the {\em frequent replacements} of methods from those pairs of
callers/callees and ranks them via a confidence metric. In contrast,
our matching principle is based on the {\em interaction similarity} of
two entities with others in the respective
versions. LibSync~\cite{oopsla10} compares two versions in which AST's
nodes are matched with both structures and names. It mines frequent
adaptation patterns via detecting frequent graph edit operations
between two usage graphs. CatchUp!~\cite{catchup05} records and stores
refactoring changes made to APIs and then replay them to adapt client
code. Diff-CatchUp~\cite{Xing2007:diffcatchup} automatically
recognizes API changes of the reused framework and suggests plausible
replacements based on working examples of the framework codebase.
Chow and Notkin~\cite{Chow1996} proposed a method for a library
maintainer annotates changed functions with rules for adaptation.

% that are used to
%generate tools that will adapt client code.


%renaming, reordering, and implementation
%changing.
%system diff, Verilog diff

%srcML, is an XML format that explicitly embeds abstract syntax within
%the source code while preserving the documentary structure as dictated
%by the developer.
%srcDiff.  The meta-difference contains specific syntactic information
%regarding the source-code changes.
%A set of program differencing tools related to {\tool} is
%semantic-based ones.

\vspace{0.05in}
\noindent {\bf Test-suite Augmentation and Adaptation.} Our tool will
complement well to existing test-suite augmentation/adaptation
tools~\cite{orso08,bert10,diffgen08}. Specifically, {\tool} could be
used first to detect the matching of classes/methods after changes are
made. Then, for each matched method/class, a test-suite augmentation
tool such as MATRIX~\cite{orso08} can be applied to detect the
fine-grained changes, to identify new/modified behaviors that are not
adequately exercised by the existing test cases, and to guide
developers in generating new test cases that exercise untested
behaviors. Another tool, DiffGen~\cite{diffgen08}, takes two versions
of a Java class, adds new branches so that behavioral differences
between the two class versions are exposed, then uses a coverage-based
test generation tool to generate test inputs for covering the added
branches. DiffGen could use the matched class/method from {\tool} as
its input.  BERT~\cite{bert10} helps developers identify regression
faults when code is changed. BERT identifies behavioral differences
between two versions through dynamic analysis by examining the changed
parts of the code and running generated tests.

\vspace{0.04in}
\noindent {\bf Program Differencing.}  Several program differencing
tools are
semantics-based~\cite{Apiwattanapong04,Horwitz1990,Jackson1994}, which
aim to detect semantic changes in a single program, rather than in
the entire system as in {\tool}. JDiff~\cite{Apiwattanapong04} detects the
differences between two versions by first matching their classes and
methods with {\em similar names/signatures}. For each pair of
name/signature-matched methods, it constructs enhanced control flow
graphs (CFGs) and matches CFGs' nodes.
%JDiff~\cite{Apiwattanapong04} detects the differences between two
%versions $P$ and $P'$. JDiff first compares each class in $P$ with the
%{\em similar-name class} in $P'$. JDiff then matches the methods in
%that class in $P$ with the {\em methods having the same
%name/signature} in the class in $P'$. For each pair of matched
%methods, it constructs enhanced control flow graphs (CFGs) and matches
%CFGs' nodes.
JDiff is still name/signature-based for entity matching, and its goal
of detecting fine-grained changes in the CFG is different than
{\tool}'s. With a similar goal of detecting fine-grained changes
within a method, Semantic diff~\cite{Jackson1994},
%Tien
Binkley {\em et al.}'s~\cite{Binkley1995},
Horwitz's~\cite{Horwitz1990}, and Raghavan {\em et al.}'s
approaches~\cite{Raghavan04} compare the corresponding program
dependence graphs in two versions. None of them considers external
interaction similarity.
%%%ASE
UMLDiff~\cite{xing05} reports design-level structural changes such as
additions, removals, relocations, attributes' modifications of
packages, classes, interfaces, methods, and fields.
and the changes to entities' dependencies.
%%%ASE
With the focus on design level, Mehra {\em et al.}~\cite{grundy05}
develop a visual differencing algorithm for design diagrams.
% in Pounamu tool. 
%%%Our tool has a different focus than design diagrams
%%%differencing.
%%% ASE

Traditional program differencing techniques (Unix's and CVS's diff
tools~\cite{Tichy1984}) are text-based in which they compare the lines
of code without considering the program's structure or semantics.
Ldiff~\cite{ldiff09} improves Unix diff by determining whether a line
is changed or is the result of additions/removals. 
Reiss' diff tool works on program token level~\cite{Reiss08}.
Other advanced diff tools perform on program's Abstract Syntax Trees
(ASTs)~\cite{Fluri07,Neamtiu05,Cottrell07}.
%%% Hunt02}.
%%%Tien: cdiff, Yang1991,
Change Distilling~\cite{Fluri07} matches two ASTs by their nodes'
names and then neighborhood's structure.  Cottrell {\em et
al.}~\cite{Cottrell07} match {\em structural} correspondences between
two classes.
srcDiff~\cite{maletic04} explicitly embeds abstract syntax within
source code with meta-difference containing syntactic information
about code changes.  {\tool}'s usage similarity is at a higher level
of semantics than those approaches.
%
{\tool} was built based on the prior work by Nguyen {\em et al.} on an
interaction-based program differencing tool,
iDiff~\cite{ase11-tool}. The key advances of {\tool} over iDiff
include 
%1) the richer set of interaction in {\tool} (Section~\ref{model}), 
1) the iterative matching in measuring the similarity of interaction
sets in Section~\ref{algosection} (iDiff does not match the entities
iteratively), 2) the application of {\tool}'s matching algorithm in
test case evolution, and 3) a new empirical evaluation on entity
matching and test case repairing recommendation.

%%%
In brief, {\tool} has key differences over those program
differencing approaches. First, they focus more on fine-grained
changes within a method, while {\tool} matches the methods/classes
across versions in an entire project.  Second, {\tool} examines not
only the internal bodies of methods for similar interactions, but also
their external interactions. 

%Finally, unlike those approaches, {\tool} considers the methods having
%same name/signature as matching candidates only, and further examines
%them for true matching via other interaction-based criteria.

\vspace{0.05in}
\noindent {\bf Refactoring Recovery.}  Several approaches have been
proposed to recover refactorings that were performed on a
program~\cite{Weissgerber06,Dig06,
Malpohl00,Antoniol04,Demeyer00,Xing07:diffcatchup}.
%
Dig {\em et al.}~\cite{Dig06} recover refactoring operations including
renamed entities in which methods with similar Shingles codes are
renamed/moved ones. 
%%Shingles is an integer encoding scheme for tokens. 
%%Among the detection of several refactorings, 
Weissgerber and Diehl~\cite{Weissgerber06} detect renamed/moved
methods by using clone detection.
% as in~\cite{Rysselberghe03}.
%Some approaches describe the changes of a program via
%operations~\cite{Asklund'04,Robbes07}.
All above approaches do not take into account the entities'
interactions.
% as in {\tool}.






%it generates a large number of test in- puts that focus on the changed
%parts of the code. Second, it runs the generated test inputs on the
%old and new versions of the code and identi?es differences in the
%tests’ behavior. Third, it analyzes the identi?ed differences and
%presents them to the developers. By fo- cusing on a subset of the code
%and leveraging differential behavior, BERT can provide developers with
%more detailed information than traditional regression testing
%approaches- approaches that rely ex- clusively on existing test
%suites

%BERT, a tool for helping developers identify regression faults that
%they may have introduced when modifying their code. BERT is based on
%the concept of behavioral regression testing: given two versions of a
%program, BERT identi?es behavioral differences between the two
%versions through dynamic analysis

%to assess the adequacy of an existing test suite,
%In previous work, we proposed MATRIX, a technique for
%test-suite augmentation based on dependence analysis and partial
%symbolic execution. In this paper, we present the next step of our
%work, where we (1) improve the effectiveness of our technique by
%identifying all relevant change-propagation paths, (2) extend the
%technique to handle multiple and more complex changes

%Given two versions of a Java class, our approach instruments the code
%by adding new branches such that if these branches can be covered by a
%test generation tool, behavioral differences between the two class
%versions are exposed. DiffGen then uses a coverage-based test
%generation tool to generate test inputs for covering the added
%branches to expose behavioral differences.


